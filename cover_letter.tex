\documentclass[a4paper,DIN]{scrlttr2}
\usepackage[pdftex]{graphicx}
\usepackage[english]{babel}
\KOMAoptions{fromphone=on,fromemail=true,backaddress=false}
\setkomavar{fromname}{Jaime E. Forero-Romero, PhD \\Assistant
  Professor}
\setkomavar{signature}{\includegraphics[scale=0.3]{jaime_firma.jpg}\\Dr. Jaime E. Forero-Romero\\Assistant Professor}
\setkomavar{fromaddress}{Physics Department, Universidad de los Andes\\
Carrera 1 18A-10, Bloque Ip. \\Bogotá - Colombia. A.A. 4976-12340.}
\setkomavar{fromemail}{je.forero at uniandes.edu.co}
\setkomavar{fromphone}{+571 3394949 Ext. 5183. Fax +571 3324516}
\setkomavar{fromfax}{s+571 3394949 Ext. 5183. Fax +571 3324516}
\setkomavar{subject}{Proposal for the creation of an Andean Regional Office of Astronomy for Development}
\begin{document}
\begin{letter}{%
Kevin Govender\\
Office of Astronomy for Development (OAD)\\
Cape Town, South Africa
}

\opening{Dear EDOC members}


The purpose of this letter is to submit a proposal for the creation of
the Andean Regional Office of Astronomy for Development (ROAD). If the
proposal is successful, we expect to have a functional node starting
on January 2015.

\

\noindent
This proposal improves upon the first proposal we submitted in
September 2013.  After its evaluation the committee suggested to clarify the
following points:

\begin{itemize}
\item The role of Chile in the Node. 
\item The objectives, leaders and budget for each one of the suggested
  projects.
\item The level of support from the institutions joining the
  Andean ROAD.
\end{itemize} 

\noindent
We have addressed these points in this proposal.

\

\noindent
The Andean ROAD will encompass six countries: Bolivia, Chile, Colombia,
Ecuador, Peru and Venezuela. The main initiative for this proposal has
been lead by several institutions in Colombia, Venezuela and
Ecuador. 

\noindent
The current state of this proposal is the result of meetings
and discussions started in 2012, continued in an Andean meeting held
in Bogot\'a in 2013 and virtual meetings during the year 2014. As a
consequence of this intensive interaction some of
the projects presented in this proposal (such as the Andean School on
Astronomy and Astrophysics) are already ongoing.


\

\noindent
In November 2013, during the XIV Latin American IAU Regional Meeting
held in Florian\'opolis (Brazil) we shared the Andean ROAD initiative
with all the attendees. This proposal also includes the input from all
the people that got interested by the Andean ROAD project. 

\

\noindent
Attached to this letter you will find the full body of the proposal
(16 pages), three support letters on behalf of the people sharing
the responsibility of coordinating the ROAD and seventeen letters from
different institutions supporting the creation of the Andean ROAD.

\

\noindent
We look forward to receiving the reply of the OAD EDOC 
to our proposal.


\closing{Sincerely,}

\vspace{-1cm}
\end{letter}

\end{document}
