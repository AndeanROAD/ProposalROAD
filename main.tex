\documentclass[12pt]{article}
\usepackage{appendix}
\usepackage[pdftex]{graphicx}
\usepackage[utf8]{inputenc}
%\usepackage[spanish]{babel}

\title{Proposal towards the establishment of an Andean Regional Office of
  Astronomy for Development} 
\begin{document}
\maketitle
\tableofcontents
\section{Rationale}

The countries in the Andean Region (Bolivia, Colombia, Chile, Ecuador,
Peru and Venezuela) represent a common language block in South
America. They share similar social conditions and goals for scientific
development.  

The astronomical development in each of these countries can be
efficiently achieved through regional cooperation. With the leadership
of Chile and Venezuela in aspects of professional astronomy and
Colombia for public outreach, it is possible to develop strategies to
strengthen the professional research, education and popularization of
astronomy in the Andean region. This development effort requires the
commitment and shared effort from different institutions. 

An Andean Regional Office of Astronomy for Development (ROAD) can
serve this goal.  It will strengthen ongoing collaboration efforts,
create channels of communication and develop new strategies to
exchange knowledge and human resources in the region.  

\section{Objectives}
\begin{itemize}
\item
Foster the goals of the International Astronomical Union (IAU) Strategic Plan in the Andean region
\item
Serve as partner to the Office of Astronomy for Development (OAD), the IAU and other international
organizations to plan and implement relevant projects in the Andean
region. 
\item
Create public forums where the project management of all development
activities can be communicated and evaluated. 
\item
Initiate and coordinate fund-raising activities for regional
development activities. 
\item
Create new institutional alliances among countries in the region to
exchange knowledge and human resources. 
\end{itemize}

\section{Core Values}
\begin{itemize}
\item 
Balance. The ROAD projects must have a balance among different areas
(research, education, outreach) and target countries. 
\item
Without borders. Priority will be given to projects that involve two
or more ROAD countries. 
\item
Transparency. All Andean ROAD projects will be transparent to the
public in their conception, execution and evaluation. 
\end{itemize}

\section{Outline of the main Andean ROAD projects}

\subsection{Task Force 1 (Universities and Research)}

\subsubsection*{Andean School on Astronomy and Astrophysics}
Every year we will hold a school aimed at advanced undergraduate
students and graduate students. The main subjects of the school have
to be broad enough allowing a large student participation. This venue
will also serve as a upstanding scenario to invite tutors from abroad
and strengthen new institutional collaborations with the Andean
ROAD. For the year 2014 the school will be hosted in Ecuador close to
the dates for the Colombian Congress of Astronomy and Astrophysics. 

\subsubsection*{Andean Peer Network}
Establishment and maintenance of an information bank of researchers in
the region. The main objective is to forgather contact information of
colleagues that can serve as project referees or partners in new
proposals. The mailing list associated to this network will also serve
as a platform to exchange information concerning fellowships,
scholarships and open positions. 

\subsubsection*{Massive Open Online Courses and the Virtual Andean
  Astronomy Seminar} 
Experiment and establish the feasibility of creating Massive Open
Online Courses at the advanced undergraduate level. In parallel, we
want to develop the infrastructure (and community interest) to hold
regular seminars to be attended by students and researchers in the
region. A requirement for this will be creation of efficient
communication channels between researchers and students. 

\subsubsection*{Andean Graduate Program}
Establish the feasibility of creating and funding a Masters course
with the objective of offering the students with an excellent
background in Astronomy and Astrophysics, while tapping on the
resources of different institutions in the region. The model for this
project is the AstroMundus program in the European Union where a
consortium of 5 Universities in 4 Countries offer a Master Program. 

\subsubsection*{Andean Postdoctoral Program}
Establish and fund a Postdoctoral program for scientists to work in
two different countries in the region for a period of 2-3 years. The
main motivation for this program is to serve as a way to help
astronomers abroad to find a permanent position in the region. 

\subsubsection*{Exploration Working Groups}
Establish working groups that will explore ways to initiate research
programs on a Andean scale. The first two suggested groups are
Astroparticle Physics (with emphasis on cosmic ray physics) and
Radioastronomy. 

\subsection{Task Force 2 (Astronomy for Children and Schools)}


\subsubsection*{Teaching Astronomy in Schools}
Creation of astronomy groups and clubs  in primary  and secondary
schools, replicating examples of projects already implemented in
Colombia, Venezuela and Chile  working together with Regional and
Education Secretaries. Such groups would be part of a network of
students  interested in astronomy, astronautics and space sciences.  

\subsubsection*{Continuous participation in  global projects} 

Such as  CERES S'COOL , Rocks around the world,  Eratosthenes, The
Galileoscope,  Dark Sky Awareness, Astronomers Without Borders, From
Earth to the Universe, International Space Week,   Noches de Galileo,
International Asteroid Search Campaign. 

\subsubsection*{Regional network of school planetariums} 

Made by the teachers attending GTTP and UNAWE workshops, replicating
experiences from UNAWE Venezuela.  

\subsubsection*{Teacher Training Projects}
UNAWE, Galileo Teacher Training Program (GTTP), Network for Astronomy
School Education (NASE) and Constellation, to strengthen teacher
training to educate in astronomy, gathering previous experiences
obtained from UNAWE, GTTP and NASE programs in the andean
countries. Local astronomers and experts from the ROAD are
invited to expand the network to more cities in other countries. Both
programs will receive support and qualification from Universities and
Research Groups active in education, astronomy and space sciences. 

\subsubsection*{Teaching resources for children and teenagers}
Continue using UNAWE methodology and resources to work with children
in schools, based on the experiences from  UNAWE Venezuela, Colombia
and Chile and build a regional network. Printed resources such as
booklets and magazines, following the example from Venezuela with a
production of over 545.000 magazines and 545.000 booklets (including a
Planisphere, Solar System Mobile, Astrorule, Spectroscopye, Solar
Clock, Quadrant, Telescope and Lunar Map) 

\subsubsection*{Olympiads and Contests}
\begin{itemize}
\item Regional, National and International Astronomy Olympiads
  (i.e. Volos, Greece, in 2013) 
\item Family, children and teenage involvement learning through games:
  Astronomy Family Marathons promoted through public institutions such
  as libraries and Education Ministries and Secretaries in the ROAD
  cities and regions.  
\end{itemize}

\subsubsection*{Inclusion programs}
New projects and teaching tools to facilitate the access of hearing
impaired and blind communities to astronomy and space sciences
topics. Ex: Astronomy for the blind, special shows for Planetariums
and puppet activities, Science in sign language experiences.   

\subsubsection*{Interdisciplinary activities}
Ethnic revisions to recover forgotten indigenous traditions through
archeoastronomy and ethnoastronomy approaches. Ex: La Chakana in
Chile, and Perú; special studies on Inca cosmovision conducted in
Perú, Bolivia, Ecuador and Venezuela; Muisca Solar Observatory in
Colombia.  

\subsection{Task Force 3 (Astronomy for the Public)}


\subsubsection*{Special shows} 

We want to develop special shows to be projected at planetariums and
science museums from the region, highlighting ancestral traditions
from indigenous tribes and founding communities from South
America. These shows would be shared between the Local and Regional
Networks of Planetariums, such as “Asociación de Planetarios del Cono
Sur”, and the growing “Red de Planetarios de Colombia”.  

\subsubsection*{Major public astronomy meetings} 

We will support participation in large astronomy events at each
country. 

\begin{itemize}
\item Stargazing events, star parties. Exchange of experiences in
the implementation, financing and evaluation of results in conducting
public events. 
\item Astronomy Family Marathons promoted through public institutions
  such as libraries and Education Ministries and Secretaries in the
  ROAD cities and regions. 

\item Celebrations. Participation and promotion of public events
  associated with astronomy global events. (Astronomy Day, Yuri's
  Night, Astronomers Without Borders, Space Week, Dark Sky Day,
  etc). Network connection in simultaneous events.  
\item Dark sky preservation campaigns.
\end{itemize}

\subsubsection*{International events}


\begin{itemize}
\item Encuentro Regional Andino de Astronomía en Isla de Pascua - 2015
\item Continuation of the projects promoted for 2014 mentioned above.
\item Communicating Astronomy for the Public - IAU - Medellín - 2015.
\item Support to the andean participation at the CAP 2015 Event to be
  held in Explora, Medellin. Attendance by representatives from each
  country to the CAP 2015.
\end{itemize}

\subsection{Joint Projects between all Task Forces} 

\subsubsection*{Common web presence}
This is a joint project of all task forces. Create a web-page that
presents the ROAD activities and future calls for
proposals \footnote{A draft webpage can be found in
  \texttt{http://comunidad.udistrital.edu.co/nodoandinodeastronomia/}}. 

\subsubsection*{Ask an astronomer}
This is a joint project of TF1 and TF3. Create a youtube channel to
receive astronomy question from all the Spanish-speaking world and
post the answers as short clips from astronomers working at
Institutions part of the Andean ROAD. 

\subsubsection*{Astronomy in the classroom}
This is a joint project of TF1 and TF2. Explore the feasibility to
establish Astronomy as part of the curricula at the school level.  

\section{Budget}

We foresee the following costs per year, one year after the
establishment of the ROAD activities.

\begin{center}
\begin{tabular}{|p{5cm} |p{6cm}|p{1.5cm}|}\hline\hline
Description & Justification&Amount (KEuro) \\\hline
TF1. Andean School & Mostly travel and accommodation for students and speakers. & 30 \\ \hline
TF1. Andean Peer Network & Initial work by programmer and
designer. Maintenance by students and postdocs.& 4 \\ \hline
TF1. MOOCs & Work by programmer, designer, video/audio. Cost
per course. & 8 \\\hline
TF1. Andean Graduate Program (Feasibility Study) &
Travel. Organization of small meetings. & 4\\\hline
TF1. Andean Postdoctoral Program & Annual postdoc salary plus research
expenses & 45 ($\times$ 5)\\ \hline
TF1. Exploration Working Groups & Travel support for each
group. Organization of small meetings. & 10
($\times$ 2)\\\hline
TF2. Regional network of school planetariums  & Material exchange. Travel.& 30\\\hline
TF2. Teacher training programs  & Travel support. Meeting organization.& 60 \\\hline
TF2. Teaching resources for children and teenagers &
Construction/Purchase. Distribution. & 60\\\hline
TF2. Olympiads and contests & Organizational costs. Travel. & 30\\\hline
TF2. Inclusion programs & Construction/Purchase of materials . Distribution.& 20\\\hline
TF3. Planetarium meeting & Travel costs & 10\\\hline
Interdisciplnary Activities & Support to different projects & 15 \\\hline
Ask an astronomer & Production of one monthly video.& 5 \\\hline
Astronomy in the classroom & Travel support. Organization of small
meetings. & 10\\\hline
Total & & 531\\\hline\hline
\end{tabular}
\end{center}

We will look for these funds with different agencies/foundations. We
will plan accordingly during the first year of activities.
\newpage.

For the first year of activities we foresee the following expenses:

\begin{center}
\begin{tabular}{|p{11cm} |p{1.5cm}|}\hline\hline
Description &Amount (Euro) \\\hline
TF1. Small grants (200 Euro each) for students that wish to travel to
another country in the Andean ROAD to pursue an internship or attend a
scientific meeting. & 1000\\\hline
TF2. Budget for the Eratostenes project, that aims to measure the
radius of the Earth with the collaboration of high-school students all
over the continent. &500\\\hline
TF2. Budget for diverse activities during 2014.&500\\\hline
TF3. Contributing budget to support an Andean meeting of
Planetaries&1000\\\hline 
Printing and distribution of posters publicizing the Andean ROAD
&350\\\hline 
Overhead costs at Universidad de los Andes over the 5000 Euro of seed
funding. & 1650\\\hline 
Coordinator’s travel to Chile and Peru to advertise the Andean ROAD
and meet with local people.& 2000\\\hline 
Total &7000\\\hline\hline
\end{tabular}
\end{center}

\noindent
5000 Euro are expected to be received by the OAD as seed funding. The
2000 Euro that cover the Coordinator’s travel to Chile and Peru will
be covered by Universidad de los Andes. 

\section{Governance}

\subsection*{Responsible Institution}
\noindent
Universidad de Los Andes (Bogotá, Colombia)

\subsection*{ROAD Coordinator}
\noindent
Mr. Jaime E. Forero-Romero, PhD\\
Assistant Professor\\
Universidad de Los Andes\\
Calle 18A \# 1 - 10\\
Physics Department, Bloque Ip, Of. 208\\
Bogot\'a, Colombia\\
Tel:  (+57 1) 339 4949, Ext. 5183\\
Fax: (+57 1) 332 4516 \\
Email: je.forero uniandes.edu.co\\

\subsection*{Oversight Committee}

The oversight committee will evalute the Andean ROAD
accomplishments. It will be composed by 4 people as follows.
\begin{enumerate}
\item OAD Representative: Kevin Govender (South Africa)
\item International TF1 Evaluator: Gustavo Bruzual (Mexico)
\item International TF1 Evaluator: Jan Tauber (Netherlands)
\item International TF2/TF3 Evaluator: Cecilia Scorza (Germany)
\end{enumerate}
\subsection*{Executive Board}
\noindent
The executive board will be composed by 4 people as follows.
\begin{enumerate}
\item ROAD Coordinator. 1 position 14\% full time. Jaime E. Forero
  Romero (Colombia) 
\item Coordinator Task Force 1. 1 position 20\% full time. Ericson
  L\'opez (Ecuador) 
\item Coordinator Task Force 2. 1 position 33\% full time. Luz Angela
  Cubides (Colombia) 
\item Coordinator Task Force 3. 1 position 33\% full time. Germán
  Puerta (Colombia) 
\end{enumerate}

\noindent
The institutional letters of support on behalf of Jaime
E. Forero-Romero and Luz Angela Cubides are attached to this
proposal. We expect that there will be support from the Bogota
Planetarium in the form of professional time up to a third of a full
time employee. It should be noted that the person at that institution
(German Puerta) who will be our main contact point already serves on
the OAD Task Force on Public Outreach so the commitment is already
present. 

\subsection*{Core functions of the Executive Board}

\begin{enumerate}
\item Decide on the acceptance of new collaborators to the Andean ROAD.
\item Create an annual action plan for the Andean ROAD.
\end{enumerate}


\subsection*{Organizing committees}


There will be three organizing committees, one for each Task
Force. The committees will have one person from each country in the
Andean ROAD. That person must belong to one of the Collaborating
Institutions. The ROAD coordinator and the specific Task Force
coordinator will also take part in the decision making process of each
committee. The committee members will be renewed every year. The
coordinators should stay in their role at most for one year and a
half. 

\subsection*{Core functions of the organizing committees}
\begin{enumerate}
\item Collect the basic information required to create annual action
  plan in line with the IAU Strategic Plan and the needs expressed by
  the Institutions composing the Andean ROAD. 
\item Decide on the priority of projects to be implemented in the
  annual action plan. 
\item Track the progress with the leaders of each project in the
  annual action plan and the IAU Strategic plan. 
\item Pursue new collaborations or organize activities with existing
  or emerging institutions both domestically (at the level of each
  ROAD’s country) and internationally. 
 \end{enumerate}

\subsection*{Andean ROAD Collaborators}


Only institutions (Universities, Research Institutes, Research
Divisions, Planetariums, Museums) or associations are considered
as collaborators of the Andean ROAD. 


Individuals can join the Andean ROAD as volunteers. In this group we
in mind four major groups. (1) undergraduate/graduate students in the
ROAD countries, (2) graduate students and postdoctoral researchers
nationals of the Andean region working abroad, (3) school teachers
working in the ROAD countries and (4) any volunteer willing to support
the goals of the Andean ROAD regardless of their country of origin. 


\subsection*{Process to join the Andean ROAD}

Institutions must direct a one page Letter of Intent (LoI) to the
members of the Executive Board stating their intention to join the
Andean ROAD and naming one (1) person of contact. Once this petition
is approved unanimously by the Executive Board, a Memorandum of
Understanding (MoU) will be signed between the new collaborating Institution
and the Responsible Institution. In order to accept the application
every TF1 institution must be partnered with a TF2/TF3 institution,
and viceversa each TF2/TF3 must have a TF1 partner in the same
country. 

There are three reasons to have these conditions. First, we want to
foster lasting processes over the ROAD lifetime; this motivates us to
have institutions, not individuals, as ROAD collaborators. Second, we want
to avoid previous experiences whereby an individual or an institution
act on a covert way on the name of whole national communities; this
motivates the two stage (Loi+MoU) process. Third, we want to motivate
direct collaboration between institutions with different backgrounds
(TF1 and TF2-TF3) to ensure a balance across task forces; this
motivates the partnering mechanism.  

Volunteers only have to express their interest to become
collaborators. They are not limited to live in the Andean ROAD
countries. The Executive Board will enable a tool to reach volunteers
as they are needed to implement different programs.  


\appendixpage
\appendix
\section{Institutions, organizations and individuals supporting this proposal}

The following institutions and organizations support this proposal and
are committed to follow the process to join the Andean ROAD. The name
of an Institution’s representative is in parenthesis. 

\subsection*{TF1 institutions and organizations}
\begin{itemize}
\item Observatorio Astronómico de Quito de la Escuela Politécnica
  Nacional, Ecuador (Dr. Ericson Lopez) 
\item Universidad Nacional de Chimborazo, Ecuador (Dr. Marlon Danilo
  Basantes Valverde) 
\item Universidad San Francisco de Quito, Ecuador (PhD Dennis Cazar
  Ramírez) 
\item Universidad de los Andes, Colombia (Dr. Jaime E. Forero-Romero) 
\item Universidad Industrial de Santander, Colombia (Dr. Luis Núñez)
\item Universidad del Valle, Colombia (Dr. Cesar A. Valenzuela-Toledo)
\item Observatorio Astronómico Universidad de Nariño, Colombia
  (MSc. James Perenguez Lopez, MSc. Karla Patricia Reyes Sánchez ) 
\item Universidad del Cauca, Colombia, (Msc. Iván Enrique Paz Narvaez) 
\item Observatorio Astronómico Nacional, Facultad de Ciencias,
  Universidad Nacional de Colombia (Prof. Giovanni Pinzón Estrada) 
\item Universidad Distrital, Bogotá Colombia (Ing. Edilberto Suárez
  Torres) 
\item Red de Estudiantes Colombianos de Astronom\'ia (Mar\'ia Camila
  Remolina Guti\'errez) 
\item Grupo Astronomía - Universidad Nacional de Ingeniería, Perú
  (Diego Berrocal Chinchay) 
\item Instituto Venezolano de Investigaciones Científicas (IVIC),
  Venezuela (Dr. Jose M Ramirez) 
\item Centro de Investigaciones de Astronomía (CIDA), Venezuela (Dra,
  Katherine Vieira) 
\item Universidad Simón Bolívar. Venezuela (Dr. Haydn Barros) 
\item Universidad Nacional Experimental del Táchira, Venezuela
  (Dr. Ramón Eveiro Molina Guillén) 
\end{itemize}

\subsection*{TF2/TF3 Institutions and organizations }
\begin{itemize}
\item Institución Armando Luna Roa, Chocó, Colombia (Esp. Martha
  Cecilia Palacios Mena) 
\item Colegio Nuestra Señora del Rosario Funza, Colombia (Profesor
  Juan Carlos Arias Cañón) 
\item Planetario de Medellín, Colombia (MSc Carlos Augusto Molina
  Velásquez, Luz Angela Cubides Gonzalez) 
\item Asociación Astronáutica Colombiana, Colombia (Aldo Esteban
  Sabogal) 
\item UEN Tibaldo Almarza Rincón, Venezuela (Lic. Lybia Mora) 
\item Agrupación de Aficionados a la Astronomía La Chakana, Chile
  (Jonathan Moncada Calabrano) 
\item Comunidad Astronómica Aficionada CAACH, Chile (Prof. Maritza
  Arias Manríquez) 
\item UNAWE-Venezuela (Enrique Torres) 
\item Universidad Pedagógica experimental Libertador, Núcleo Monagas,
  Venezuela (Profesor Freddy Oropeza) 
\end{itemize}

\subsection*{Individuals supporting this proposal}

\begin{itemize}
\item Dr. Jan Tauber, European Space Agency, Scientific Support
  Office, The Netherlands 
\item Dr. Cecilia Scorza, Haus der Astronomie, Germany
\item Dr. Antonio Pereyra, Instituto Nacional de Pesquisas Espaciais
  (INPE), Brasil 
\item Rosario Moyano Aguirre, Astronomía Sigma Octante Bolivia 
\item Dr. Nicolás Vásquez, Escuela Politécnica Nacional, Ecuador
\item Miguel Garcia, MSc, Escuela Politécnica Nacional, Ecuador
\item Phd Mario Armando Higuera, Observatorio Astronómico Nacional,
  Universidad Nacional de Colombia 
\item Ángela Patricia Pérez Henao, Astronomy Kids Club y Planetario de
  Bogotá, Colombia 
\item Juan Sebastián Florez Suancha, Universidad de Antioquia, Colombia
\item León J. Restrepo Quirós , Universidad de San Buenaventura/GTTP
  Colombia, Colombia 
\item Dr. Rigoberto Casas Miranda, Universidad Nacional de Colombia, Colombia
\item Dr. José Alejandro García Varela, Universidad de los Andes,
  Departamento de Física, Colombia 
\item Dr. Beatriz Eugenia Sabogal Martínez, Universidad de los Andes,
  Departamento de Física, Colombia 
\item Diana Milena Navarro, IED Fernando Mazuera, Colombia
Cristian Góez Therán, Olimpiadas Colombianas de Astronomía y
Astrofísica, Colombia 
\item Ing Edgar Quintanilla Piña, Universidad Industrial de
  Santander. Grupo Halley Sede Socorro. Colombia 
\item Malory Agudelo Vásquez, Universidad de Antioquia, Colombia
\item Juan Carlos Beamín M., Instituto Astrofísica, Pontificia
  Universidad Católica de Chile 
\end{itemize}


\section{Brief description of the coordinating institution}

Universidad de los Andes (Uniandes) is located in Bogotá, Colombia. It
is a private University founded in 1946. Currently it is the top
University in Colombia as reported by the QS 2013-2014 and Times
Higher Education 2013 University Rankings. It also ranks among the top 4 in
Latin America and the top 300 on a global scale.   

Uniandes hosts a Faculty of Sciences with 5 departments: Biology,
Chemistry, Geosciences, Mathematics and Physics. The Physics
Department has an active graduate school and also hosts postdoctoral
researchers. One of the research groups in the Physics is focused on
Astronomy and Astrophysics, composed by 4 Faculty (3 PhD, 1 MSc) with
research interests in instrumentation, observational astronomy and
computational astrophysics. The Astrophysics Group also manages a
small observatory dedicated mostly to outreach and motivating
undergraduate students into astronomy. 

In July 2013 Universidad de los Andes hosted and gave finantial
support to the Workshop Astronomía en los
Andes \footnote{\texttt{http://workshopastronomia.uniandes.edu.co/}}. This
Workshop received researchers from all the countries participating in
this ROAD proposal and also included the participation of the TF2 and
TF3 coordinators in the current proposal. This meeting helped us to
survey the status of Astronomy in the region and design a blueprint
for its development in the region. Submitting the current proposal was
one of the main goals of the workshop. 
 

\section{Names and affiliations of the committee members for the year 2014}

\subsection*{TF1 Committee}
\begin{itemize}
\item Nicol\'as Vasquez (EPN, Ecuador)
\item Luis Otiniano (CONIDA, Peru)
\item Giovanni Pinz\'on (OAN, Colombia)
\item Mirko Raljevic (UMSA, Bolivia)
\item Eduardo Unda-Sanzana (SOCHIAS, Chile)
\item Kathy Vieira (CIDA, Venezuela)
\end{itemize}

\subsection*{TF2 Committee}
\begin{itemize}
\item Angela Patricia Pérez (UNAWE, Colombia)
\item Enrique Torres  (IVIC, Venezuela)
\item Manuel de la Torre (Olimpiadas de Astronomía, Bolivia) 
\item Javier Ramirez (Planetario de Lima, Perú)
\end{itemize}

Supporting committee (Mauricio Chacón (Colombia), Leonardo Ariza
(Colombia), Juan Carlos Arias (Colombia), Alvaro Cano (Colombia), León
Restrepo (Colombia), Ymmer Vanegas (Venezuela), Freddy Oropeza
(Venezuela), Rosario Moyano (Bolivia), Angel Carranza (Perú), Fernando
Camacho (Perú), Maritza Arias (Chile), Maria Paz Cornejo (Chile),
Johnatan Moncada (Chile)) \footnote{The mentioned coordinators have
  been proposed by the participants of the weekly virtual meetings
  held since April 2013 taking into account their networking skills
  and compromise, and are supported by the committee members mentioned
  below. As of the end of September 2013, there have been
  representatives from Venezuela, Perú, Bolivia, Chile and Colombia
  constantly active during the TF2 meetings, but so far there has been
  no participation from the TF2 community in Ecuador. } 



\subsection*{TF3 Committee}
\begin{itemize}
\item Carlos Molina (Colombia)
\item Johnny Cova (Venezuela)
\end{itemize}

Supporting committee (Cristian Goez (Colombia), Carlos Quintana (Venezuela))

NB. Due to the large size of the TF2 and TF3 community in Chile, the
process to select their representatives in the corresponding
committees will be completed by 2014. 
 
\end{document}
