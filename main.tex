\documentclass[12pt]{article}
\usepackage{appendix}
\usepackage[pdftex]{graphicx}
\usepackage[utf8]{inputenc}
\usepackage[colorlinks=true,urlcolor=blue,linkcolor=blue]{hyperref} % navega por el doc
%\usepackage[spanish]{babel}

\title{Proposal towards the establishment of an Andean Regional Office of
  Astronomy for Development} 
\date{November 1, 2014}
\begin{document}
\maketitle
\tableofcontents
\newpage

\section{Rationale}

The countries in the Andean Region (Bolivia, Colombia, Chile, Ecuador,
Peru and Venezuela) represent a common language block in South
America. They share similar social conditions and goals for scientific
and cultural development.  

The astronomical development in each of these countries can be
efficiently achieved through regional cooperation. With the leadership
of Chile and Venezuela in aspects of professional astronomy and
Colombia for public outreach, it is possible to develop strategies to
strengthen the professional research, education and popularization of
astronomy in the Andean region. This development effort requires the
commitment and shared effort from different institutions. 

An Andean Regional Office of Astronomy for Development (ROAD) can
serve this goal.  It will strengthen ongoing collaboration efforts,
create channels of communication and develop new strategies to
exchange knowledge and human resources in the region.  


The participating institutions have been involved in different
projects in each one of the ROAD core areas: research, schools and
children and public \footnote{For instance, in the outreach and children activities
  the following programs have been active during the last years: Astronomy Olympiads, Galileo Teacher Training
  Program (GTTP), the Network for Astronomy School Education (NASE)
  and Universe Awareness (UNAWE), Galileo Mobile, CERE S'COOL, Rocks
  Around the World, Eratosthenes, Galileoscope, Dark Sky Awareness,
  Astronomers without borders, From Earth to the Universe,
  International Space Week, Galileo Nights and Asteroid Search
  Campaigns.}. However, the participation and degree of commitment
has not been homogeneous across the region, or even inside the same
country. 

Therefore our {\bf main goal is to guarantee and strengthen effective
  methods of communication between the representatives and
  coordinators of global, regional and local projects} implemented  in
the Andean countries, specially looking to orient and advise new
working groups in  other cities and regions. This lead us to define
the following objectives.


\section{Objectives}
\begin{itemize}
\item
Foster the goals of the International Astronomical Union (IAU) Strategic Plan in the Andean region
\item
Serve as partner to the Office of Astronomy for Development (OAD), the IAU and other international
organizations to plan and implement relevant projects in the Andean
region. 
\item
Create public forums where the project management of all development
activities can be communicated and evaluated. 
\item
Initiate and coordinate fund-raising activities for regional
development activities. 
\item
Create new institutional alliances among countries in the region to
exchange knowledge and human resources. 
\end{itemize}

\section{Core Values}
\begin{itemize}
\item 
Balance. The ROAD projects must have a balance among different areas
(research, education, outreach) and target countries. 
\item
Without borders. Priority will be given to projects that involve two
or more ROAD countries. 
\item
Transparency. All Andean ROAD projects will be transparent to the
public in their conception, execution and evaluation. 
\end{itemize}

\section{Institutions Supporting this Proposal}

\begin{itemize}
\item {Universidad Mayor de San Sim\'on (Bolivia).}
\item {Sociedad Chilena de Astronom\'ia, SOCHIAS (Chile).}
\item {Escuela Colombiana de Carreras Industriales (Colombia).}
\item {Observatorio Astron\'omico Nacional (Colombia).}
\item {Planetario de Bogot\'a (Colombia).}
\item {Parque Explora - Planetario de Medell\'in (Colombia).}
\item {Universidad Industrial de Santander (Colombia).}
\item {Universidad Sergio Arboleda (Colombia).}
\item {Universidad de Antioquia (Colombia).}
\item {Universidad del Valle (Colombia).}
\item {Universidad Distrital Francisco Jose de Caldas(Colombia).}
\item {Universidad de los Andes (Colombia).}
\item {Escuela Polit\'ecnica Nacional (Ecuador).}
\item {Universidad San Francisco de Quito (Ecuador).}
\item {Instituto Geof\'isico del Per\'u (Peru).}
\item {Universidad Nacional de Ingenier\'ia (Peru).}
\item {Centro de Investigaciones de Astronomia, CIDA (Venezuela).}
\item {Instituto Venezolano de Investigaciones Cient\'ificas, IVIC (Venezuela)}
\end{itemize}




\section{Outline of the main Andean ROAD projects}

\subsection{Task Force 1 (Universities and Research)}

\subsubsection{Andean School on Astronomy and Astrophysics}

\begin{itemize}

\item{\bf Objective}: Organize a school for advanced undergraduate
  students and graduate students.
\item{\bf Leading ROAD Institution}: Changes every year.
\item{\bf Contact}: Changes every year.
\item{\bf Implementation}:
Every year we will hold a school aimed at
  advanced undergraduate students and graduate students. The main
  subjects of the school have to be broad enough allowing a large
  student participation. This venue will also serve as a upstanding
  scenario to invite tutors from abroad and strengthen new
  institutional collaborations with the Andean ROAD. The first school
  will take place in 2014 hosted in Ecuador close to the dates for the
  Colombian Congress of Astronomy and Astrophysics.  
\item{\bf Budget}: 10KEuro/year
\end{itemize}


\subsubsection{Andean Graduate Program}
\begin{itemize}
\item{\bf Objective}: Determine the feasibility of creating and
  funding an Andean graduate program. 
\item{\bf Leading ROAD Institutions}: Observatorio Astron\'omico
  Nacional (Colombia); SOCHIAS ROAD office  (Chile, Oficina
  Nacional de Coordinaci\'on, ONC);  Universidad San
  Francisco de Quito (Ecuador).
\item{\bf Contacts}: Giovanni Pinz\'on, PhD; Eduardo Unda-Sanzana,
  PhD; Dennis Cazar Ram\'irez, PhD. 
\item{\bf Implementation}:
  Explore the possibility to coordinate the use of
  educational resources of different institutions in the region. The
  model for this project is the AstroMundus program in the European
  Union where a consortium of 5 Universities in 4 Countries offer a
  Master Program.  This research will be done mainly thrugh virtual
  meetings.  
\item{\bf Budget}: 0 KEuro/year.
\end{itemize}

\subsubsection{Exploration Working Group in Astroparticle Physics}
\begin{itemize}
\item{\bf Objective}: Establish a cooperation network of institutions
  interested in Astroparticle Physics.
\item{\bf Leading ROAD Institution}: Universidad
  Industrial de Santander (Colombia); Universidad San
  Francisco de Quito (Ecuador). 
\item{\bf Contact}: Luis Nu\~nez, PhD; Dennis Cazar Ram\'irez, PhD. 
\item{\bf Implementation}: The leading institutions will organize
  periodic meetings to gather all the groups in the Andean Region
  interested in developing its research capabilities in astroparticle
  physics. This will be done through workshops and the installations
  of small research stations for the Latin American Giant Observatory
  (LAGO) project, which has already been kick-started 
  in Venezuela, Colombia, Ecuador, Peru and Bolivia.
\item{\bf Budget}: 10KEuro/year.
\end{itemize}

\subsubsection{Exploration Working Group in Radioastronomy}
\begin{itemize}
\item{\bf Objective}: Establish a cooperation network of institutions interested in the development of Radioastronomy.
\item{\bf Leading ROAD Institution:} Escuela Colombiana de Carreas
  Industriales (Colombia).
\item{\bf Contact}: Germ\'an Chaparro Molano, PhD.
\item{\bf Implementation}: The Leading Institution will organize
  periodic meetings to gather all the groups in the Andean Region
  interested in the development of Radioastrony capabilities. The
  first meeting is planned for 2015 
The Leading Institution will also organize a Training School to build
up new capabilities and strengthen the ties between different groups
and countries. The first school is planned for 2016.
\item{\bf Budget}: 10KEuro/year.
\end{itemize}


\subsubsection{Massive Open Online Courses}
\begin{itemize}
\item{\bf Objective}: Create a Massive Open
Online Course at the introductory undergraduate level with sections for
the general public. 
\item{\bf Leading ROAD Institution}: Universidad Industrial de
  Santander (Colombia).  
\item{\bf Contact}: Luis Nu\~nez, PhD
\item{\bf Implementation}: 
We will create a Massive Online Open
Course in Astronomy and Astrophysics, that could be used either as a
general course for a wide audience or as an introductory course for
undergraduate students of  Physics programs. Planned for 14 weeks,
the course will contain four major blocks: Astronomy and celestial
bodies, Instrumentation in Astronomy, Current Trends in Astronomy and
Astrophysics, and  Data Bases and Data analysis in Astronomy. The
first three blocks will be divided in two modules corresponding to two
different levels: A and B. General audience could follow the level A
for the three modules and physics students should follow whole content
of the course. Each module, will have a 15m video, recommended reading
and several assignments.  
\item{\bf Budget}: 10KEuro/Year.
\end{itemize}


\subsubsection{Communication Network}
\begin{itemize}
  \item{\bf Objective}: Gather contact information of all TF1 colleagues in
    the ROAD.
  \item{\bf Leading ROAD Institution}: Universidad de los Andes (Colombia)
  \item{\bf Contact}: Jaime E. Forero-Romero, PhD
  \item{\bf Implementation:} Open a mailing list associated to TF1
    activities. This network will serve as a platform to exchange
    information concerning joint projects, fellowships, scholarships
    and open positions. There will also be a central web-page for the
    Andean ROAD showing the current ongoing projects and advances.
  \item{\bf Budget}: 0KEuro/year.
\end{itemize}


\subsection{Task Force 2 (Astronomy for Children and Schools)}


\subsubsection{Periodic coordination sessions}

\begin{itemize}

\item{\bf Objective}: Strengthen communication  strategies between
  teachers  and coordinators from each  country
\item{\bf Leading ROAD Institution}: Parque EXPLORA (Colombia), IVIC
  (Venezuela). 
\item{\bf Contact}: Luz Angela Cubides,  Angela Patricia  P\'erez,
  Enrique  Torres.
\item{\bf Implementation}: Hold bi-weekly coordination sessions. They
  have been held during 2014.
\item{\bf Budget}: 0Euro/year.
\end{itemize}


\subsubsection{Virtual training sessions}
\begin{itemize}
\item {\bf Objective}: Create virtual training and activities open to
  teachers and students. 
\item {\bf Leading ROAD Institution}: IVIC (Venezuela).
\item {\bf Contact}: Enrique Torres, Juan Carlos Arias.
\item {\bf Implementation}. Develop a virtual platform to hold virtual
  workshops on information technologies applied to astronomy and space
  sciences.
\item {\bf Budget}: 1 KEuro/year.
\end{itemize}

\subsubsection{Teaching material for visually impaired students}
\begin{itemize}
\item {\bf Objective}: Design teaching material to work with visually
  impaired students. 
\item {\bf Leading ROAD Institution}: Planetario de Bogot\'a
  (Colombia). 
\item {\bf Contact}: Angela Patricia Perez, Dilia Gonzalez. 
\item {\bf Implementation}: We start by doing research on the kind of
  material to develop. Later on we will define the production and
  distribution processes.  
\item {\bf Budget}: 1 KEuro/Year
\end{itemize}

\subsubsection{Annual TF2 meeting}
\begin{itemize}
\item {\bf Objective}:  Organize and annual meeting to gather all the
  collaborators in the Task Force.
\item {\bf Leading ROAD Institution}: Changes every year.
\item {\bf Contact}: Maritza Arias Manriquez, Jonathan Moncada,  Manuel
  de la Torre.
\item {\bf Implementation}: 
Every year we will hold a meeting aimed at collaborators in the
node. The venue will change every year. The first meeting will take
place in 2015 with the theme Archeoastronomy.
\item {\bf Budget}: 10 KEuro/year.
\end{itemize}




\subsection{Task Force 3 (Astronomy for the Public)}


\subsubsection{Development for planetariums}
\begin{itemize}
\item {\bf Objective}: Develop special shows for planetariums in the
  region. 
\item {\bf Leading ROAD Institution}: Parque EXPLORA (Colombia).
\item {\bf Contact}: Carlos Molina.
\item {\bf Implementation}: We want to develop special shows to be projected
  at planetariums and science museums from the region, highlighting
  ancestral traditions from indigenous tribes and founding communities
  from South America. These shows would be shared between the Local
  and Regional Networks of Planetariums, such as “Asociación de
  Planetarios del Cono Sur”, and the growing “Red de Planetarios de
  Colombia”.   
\item {\bf Budget}: 10 KEuro/year.
\end{itemize}


\subsubsection{Communicating astronomy with the
  public.}
\begin{itemize}
\item {\bf Objective}: Organize periodic meetings to gather all the
  collaborators in the Task Force, "Communicating Astronomy with the
  Public".  
\item {\bf Leading ROAD Institution}: SOCHIAS ROAD office
 (Chile, Oficina
  Nacional de Coordinaci\'on, ONC).
\item {\bf Contact}: Farid Char.
\item {\bf Implementation}: We want to hold regular meetings that can
  serve as an exchange platform for people involved in communicating
  astronomy to the public; this includes science museums,
  planetariums, mass media and associated industries. We also want to use
  this opportunity to showcase efforts in the other two task
  forces. This meeting will be held parallel to large professional
  meetings. We envision this event as a "Communicating Astronomy
  with the
  Public" \footnote{\url{http://www.communicatingastronomy.org/}. The
    next global meeting for this event will be held in 2016 in Colombia.}
  meeting for the region. We expect to hold the first one during the
  next Latin-American Regional IAU Meeting in 2016.  
\item {\bf Budget}: 10 KEuro/year.
\end{itemize}




\section{Budget}

We foresee the following costs per year in each task force. 
\begin{itemize}
\item Task Force 1: 40 KEuro/year
\item Task Force 2: 12 KEuro/year
\item Task Force 2: 20 KEuro/year
\end{itemize}

The costs are covered by in-kind contributions and proposals
submitted for each different project according to its needs.

The seed-funding (5 KEuro) provided by the ROAD central office will be used as follows.

\begin{center}
\begin{tabular}{|p{11cm} |p{1.5cm}|}\hline\hline
Description &Amount (Euro) \\\hline
TF1. Small grants (200 Euro each) for students that wish to attend the
Andean School for Astronomy and Astrophysics. & 1000\\\hline 
TF2.  Teaching material for visually impaired students & 1000\\\hline
TF3. Small grants (250 Euro each) for outreach professionals that
wish to attend the meeting "Communicating Astronomy with the Public".&
1000\\\hline  
Printing and distribution of posters publicizing the Andean ROAD
&400 \\\hline 
Overhead costs at Universidad de los Andes over the 5000 Euro of seed
funding. & 1600\\\hline 
Total &5000\\\hline\hline
\end{tabular}
\end{center}


\section{Governance}

\subsection*{Responsible Institution}
\noindent
Universidad de Los Andes (Bogotá, Colombia)

\subsection*{ROAD Coordinator}
\noindent
Mr. Jaime E. Forero-Romero, PhD\\
Assistant Professor\\
Universidad de Los Andes\\
Calle 18A \# 1 - 10\\
Physics Department, Bloque Ip, Of. 208\\
Bogot\'a, Colombia\\
Tel:  (+57 1) 339 4949, Ext. 5183\\
Fax: (+57 1) 332 4516 \\
Email: je.forero uniandes.edu.co\\

\subsection*{Oversight Committee}

The oversight committee will evaluate the Andean ROAD
accomplishments. It will be composed by 4 people as follows.
\begin{enumerate}
\item OAD Representative: Kevin Govender (South Africa)
\item International TF1 Evaluator: Gustavo Bruzual (Mexico)
\item International TF1 Evaluator: Jan Tauber (Netherlands)
\item International TF2/TF3 Evaluator: Cecilia Scorza (Germany)
\end{enumerate}
\subsection*{Executive Board}
\noindent
The executive board will be composed by 3 people as follows.
\begin{enumerate}
\item ROAD Coordinator/ Coordinator Task Force 1. 1 position 20\% full
  time. Jaime E. Forero-Romero (Colombia) 
\item Coordinator Task Force 2. 1 position 30\% full time. Luz Angela
  Cubides (Colombia) 
\item Coordinator Task Force 3. 1 position 50\% full time. Farid Char
  (Chile). 
\end{enumerate}

\noindent
The institutional letters of support on behalf of Jaime
E. Forero-Romero (Universidad de los Andes), Luz Angela Cubides
(Parque Explora-Planetario de Medell\'in) and Farid Char (SOCHIAS) are
attached to this proposal. 


\subsection*{Core functions of the Executive Board}

\begin{enumerate}
\item Decide on the acceptance of new collaborators to the Andean ROAD.
\item Create an annual action plan for the Andean ROAD.
\end{enumerate}


\subsection*{Organizing committees}

There will be three organizing committees, one for each Task
Force. We aim to have one person from each country in the
Andean ROAD. Members in that committee must belong to one of the
Collaborating Institutions. The ROAD coordinator and the specific Task
Force coordinator will also take part in the decision making process of each
committee. The committee members will be renewed every year. The
coordinators should stay in their role at most for one year and a
half. 

\subsection*{Core functions of the organizing committees}
\begin{enumerate}
\item Collect the basic information required to create annual action
  plan in line with the IAU Strategic Plan and the needs expressed by
  the Institutions composing the Andean ROAD. 
\item Decide on the priority of projects to be implemented in the
  annual action plan. 
\item Track the progress with the leaders of each project in the
  annual action plan and the IAU Strategic plan. 
\item Pursue new collaborations or organize activities with existing
  or emerging institutions both domestically (at the level of each
  ROAD’s country) and internationally. 
 \end{enumerate}

\subsection*{Andean ROAD Collaborators}


Only institutions (Universities, Research Institutes, Research
Divisions, Planetariums, Museums) or professional associations are
considered as collaborators of the Andean ROAD. 

Individuals can join the Andean ROAD as volunteers. In this group we
in mind four major groups. (1) undergraduate/graduate students in the
ROAD countries, (2) graduate students and postdoctoral researchers
nationals of the Andean region working abroad, (3) school teachers
working in the ROAD countries and (4) any volunteer willing to support
the goals of the Andean ROAD regardless of their country of origin. 


\subsection*{Process to join the Andean ROAD}

Institutions must direct a one page Letter of Intent (LoI) to the
members of the Executive Board stating their intention to join the
Andean ROAD and naming one (1) person of contact. Once this petition
is approved unanimously by the Executive Board, a Memorandum of
Understanding (MoU) will be signed between the new collaborating Institution
and the Responsible Institution. In order to accept the application
every TF1 institution must have TF2/TF3 partner institution,
and viceversa each TF2/TF3 must have a TF1 partner in the same
country. 

There are three reasons to have these conditions. First, we want to
foster lasting processes over the ROAD lifetime; this motivates us to
have institutions, not individuals, as ROAD collaborators. Second, we want
to avoid previous experiences whereby an individual or an institution
act on a covert way on the name of whole national communities; this
motivates the two stage (Loi+MoU) process. Third, we want to motivate
direct collaboration between institutions with different backgrounds
(TF1 and TF2-TF3) to ensure a balance across task forces; this
motivates the partnering mechanism.  

Volunteers only have to express their interest to become
collaborators. They are not limited to live in the Andean ROAD
countries. The Executive Board will enable a tool to reach volunteers
as they are needed to implement different programs.  

\subsection*{The role of Chile in the Andean ROAD}

Due to the large size of the Chilean professional astronomical
community and its special role in global astronomy its important to
spell out in detail the role of Chile in the collaboration

The most important aspect is that all the interaction to coordinate
activities with Chile will be done through the Sociedad Chilena de
Astronomia (SOCHIAS) --- the society that gathers all the professional
astronomers in the country.

SOCHIAS has designated Eduardo Unda-Sanzana as the main contact
person for TF1 tasks, Cristian Cort\'es for TF2 tasks and Patricio
Rojo for TF3 tasks inside Chile. Farid Char is a staff member at the
Unidad de Astronomía of the Universidad de Antofagasta who will be
responsible to interface with the rest of the ROAD countries. 

Furthermore, we have been in touch with Ms. Valentina Rodriguez
(currently employed at ESO in Chile) who is leading an effort to create
an office dedicated to stablish relationships with {\bf{the
  governement and industry in Chile}} on themes directly connected to
Astronomy for Development. This could be considered as a
Fourth Task Force based entirely in Chile. They are willing to
back the efforts of the Andean ROAD as they write in the attached
support letter signed by Valentina Rodriguez, Dr. Guillermo Blanc,
Dr. Eduardo Bendek. Other people that know about this initiative (althought they
did not have time to sign the letter) are Prof. Massimo Tarenghi (ESO),
Dr. Jorge Ibsen (ALMA) and Paulina Bocaz (NRAO).  

\appendixpage
\appendix

\section{Brief description of the coordinating institution}

Universidad de los Andes (Uniandes) is located in Bogotá, Colombia. It
is a private University founded in 1946. Currently it is the top
University in Colombia as reported by the QS2014 and Times
Higher Education 2013-2014 University Rankings. It also ranks among
the top 5 in Latin America and the top 300 on a global scale.    

Uniandes hosts a Faculty of Sciences with 5 departments: Biology,
Chemistry, Geosciences, Mathematics and Physics. The Physics
Department has an active graduate school and also hosts postdoctoral
researchers. One of the research groups in the Physics is focused on
Astronomy and Astrophysics, composed by 4 Faculty (3 PhD, 1 MSc) with
research interests in instrumentation, observational astronomy and
computational astrophysics. The Astrophysics Group also manages a
small observatory dedicated mostly to outreach and motivating
undergraduate students into astronomy. 

In July 2013 Universidad de los Andes hosted and gave financial
support to the Workshop Astronomía en los
Andes \footnote{\texttt{http://workshopastronomia.uniandes.edu.co/}}. This
Workshop received researchers from all the countries participating in
this ROAD proposal and also included the participation of the TF2 and
TF3 coordinators in the current proposal. This meeting helped us to
survey the status of Astronomy in the region and design a blueprint
for its development in the region. Submitting the current proposal was
one of the main goals of the workshop. 
 
\newpage



\section{Names and affiliations of the committee members for the year 2015}

\subsection*{TF1 Committee}
\begin{itemize}
\item Nicolas V\'asquez (EPN, Ecuador)
\item Giovanni Pinz\'on (OAN, Colombia)
\item Eduardo Unda-Sanzana (SOCHIAS, Chile)
\item Kathy Vieira (CIDA, Venezuela)
\end{itemize}

\subsection*{TF2 Committee}
\begin{itemize}
\item Enrique Torres  (IVIC, Venezuela)
\item Cristian Cort\'es (SOCHIAS, Chile)
\end{itemize}

\subsection*{TF3 Committee}
\begin{itemize}
\item Carlos Molina (Parque Explora, Colombia)
\item Johnny Cova (CIDA, Venezuela)
\item Patricio Rojo (SOCHIAS, Chile)
\end{itemize}

\newpage
\section{Support letters for the coordinators}

\newpage
\section{Letters from the institutions supporting the Andean ROAD}

\end{document}
